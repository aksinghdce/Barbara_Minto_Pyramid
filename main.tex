\documentclass[11pt, a4paper, margin=1in]{IEEEtran}
\usepackage{color}

%%%%%%%%%%%%%%%%%%%%%%%%%%%%%%%%%%%%%%%%%%%%%Timeline%%%%%%%%%%%%
\usepackage[margin=3cm]{geometry}
\usepackage{ragged2e}
\usepackage{fourier}
\usepackage{tikz} 
\usetikzlibrary{chains,shapes.arrows,fit}

\definecolor{arrowcolor}{RGB}{201,216,232}% color for the arrow filling
\definecolor{circlecolor}{RGB}{79,129,189}% color for the inner circles filling
\colorlet{textcolor}{white}% color for the text inside the circles
\colorlet{bordercolor}{white}% color for the outer border of circles

\pgfdeclarelayer{background}
\pgfsetlayers{background,main}

\newcounter{task}

\newlength\taskwidth% width of the box for the task description
\newlength\taskvsep% vertical distance between the task description and arrow

\setlength\taskwidth{2.5cm}
\setlength\taskvsep{17pt}

\def\taskpos{}
\def\taskanchor{}

\newcommand\task[1]{%
  {\parbox[t]{\taskwidth}{\scriptsize\Centering#1}}}

\tikzset{
inner/.style={
  on chain,
  circle,
  inner sep=4pt,
  fill=circlecolor,
  line width=1.5pt,
  draw=bordercolor,
  text width=1.2em,
  align=center,
  text height=1.25ex,
  text depth=0ex
},
on grid
}

\newcommand\Task[2][]{%
\node[inner xsep=0pt] (c1) {\phantom{A}};
\stepcounter{task}
\ifodd\thetask\relax
  \renewcommand\taskpos{\taskvsep}\renewcommand\taskanchor{south}
\else
  \renewcommand\taskpos{-\taskvsep}\renewcommand\taskanchor{north}
\fi
\node[inner,font=\footnotesize\sffamily\color{textcolor}]    
  (c\the\numexpr\value{task}+1\relax) {#1};
\node[anchor=\taskanchor,yshift=\taskpos] 
  at (c\the\numexpr\value{task}+1\relax) {\task{#2}};
}

\newcommand\drawarrow{% the arrow is placed in the background layer 
                                                     % after the node for the tasks have been placed
\ifnum\thetask=0\relax
  \node[on chain] (c1) {}; % if no \Task command is used, the arrow will be drawn
\fi
\node[on chain] (f) {};
\begin{pgfonlayer}{background}
\node[
  inner sep=10pt,
  single arrow,
  single arrow head extend=0.8cm,
  draw=none,
  fill=arrowcolor,
  fit= (c1) (f)
] (arrow) {};
\fill[white] % the decoration at the tail of the arrow
  (arrow.before tail) -- (c1|-arrow.west) -- (arrow.after tail) -- cycle;
\end{pgfonlayer}
}

\newenvironment{timeline}[1][node distance=.75\taskwidth]
  {\par\noindent\begin{tikzpicture}[start chain,#1]}
  {\drawarrow\end{tikzpicture}\par}
%%%%%%%%%%%%%%%%%%%%%%%%%%%%%%%%%%%%%%%%%%%%%%%%%%%%%%%%%%%%%%%%%

\begin{document}
% Top Matter
\title{I Say: Don't Bully The Nature}
\author{Amit Kumar Singh}
\maketitle

\renewcommand{\abstractname}{Executive Summary}
\begin{abstract}
This article attempts to estimate the \textbf{goal}, \textbf{short-term actions} and \textbf{long-term strategies} that Marianne Barner \textbf{needs to decide} to deal with a continuing problem that reveals iteself with respect to IKEA's code of conduct (IWAY).
\end{abstract}
\section{Situation}
Since 1993, when "Child Labor Deterrance Act" was proposed in US Congress, IKEA has been setting up \textbf{policies and practices} to closely \textbf{monitor its suppliers}. The efforts of IKEA has lead to its redefined relationship with its suppliers. According to reports, \textbf{benefits and concerns} associated with the aforementioned policies and practices, are fundamental to IKEA's business, in particular, and with international socioeconomic structure, in general.
\section{Complications}
"central premise is that the total amount of control people are subjected to, relative to the control they can exercise, will affect the probability and type of their deviant behavior." \cite{tittle2018control}
\section{Questions}
I question IKEA's decision to reject Rugmark Consortium's invitation to join the Consortium. Given the fact that it was the best option of that time. \cite{publichearing1998}

"..the success or failure of early nineteenth-century child labor laws depended on these actors’ social skill, pragmatic creativity, and goal-directedness." \cite{anderson2018policy} Here "actors" refers to "elite policy entrepreneurs".

\section{Appendix}
\subsection{Timeline}
\section{Appendix A}

I present a timeline in "Figure 1", which shows chronological events in the case. I also show the \%-change in sales YoY (in "Figure 2"). 

By observing these diagrams in juxtaposition one can observe relationship between the geopolitical events mentioned in the timeline and dip in \%-change in sales. Between the years 1995 to year 2000 large amount of change in sales is observed and during the same time frame, as observed in the timeline from circle marked to that marked 6, one can observe negative coverage in media were reported. This shows the interdependence of consumer perception of a brand, the media agencies that influence this perception and phenomenon that how these factors interact with a business's performance. \\
\subsection{Timeline}
Figure 1: The Chronology of Events for IKEA's Child-Labor Ordeal \\

\begin{timeline}
\Task[1]{\textbf{Seed:Child Labor Deterrent Act} was proposed for debate in U.S. Congress\\\textbf{Year 1993} }
\Task[1.1]{\textbf{Unintended Consequences}}
\Task[2]{\textbf{Problem:}First report of Child Labor\\\textbf{Mitigation}\\Strategy:International Labor Organization (ILO)\\\textbf{Year 1994}}
\Task[2.1]{\textbf{Unintended Consequences 2} Induced unemployment in Pakistan}
\Task[3]{\textbf{Problem:}Monitoring was impractical\\\textbf{Scale and Nature of Manufacturing Process}\\\textbf{Fall 1994}}
\Task[4]{\textbf{May 1995}\\\textbf{Problem:}Investigative Report of Child Labor\\\textbf{Mitigation}\\Evidence:Report found fabricated}
\Task[5]{\textbf{Year 1998}\\Marianne Barner becomes IKEA\textquotesingle s children\textquotesingle s ombudsman}
\end{timeline}

\begin{timeline}
\Task[6]{\textbf{1998-99} 10-day educational tours of India\\ for IKEA's retail managers from different countries}
\Task[7]{\textbf{2004} registered success of joint program by UNICEF and IKEA\\ with ACL and Self-Help groups programs}
\Task[7.1]{\textbf{2004} IKEA terminated 10 supplier relationships for violations of its IWAY Code}
\Task[8]{\textbf{2005} Dahlvig's \textbf{10 Jobs for 10 Years} document released}
\end{timeline}
\subsection{References}
\begin{thebibliography}{9}
    \bibitem{tittle2018control}
    Tittle, C. R. (2018). Control balance: Toward a general theory of deviance. Routledge.
    \bibitem{anderson2018policy}
    Anderson, E. (2018). Policy Entrepreneurs and the Origins of the Regulatory Welfare State: Child Labor Reform in Nineteenth-Century Europe. American Sociological Review, 83(1), 173-211.
    \bibitem{publichearing1998}
    United States. International Child Labor Program. (1998). Public hearings on international child labor. United States:Page:183
\end{thebibliography}
\end{document}