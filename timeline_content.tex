\section{Appendix A}

I present a timeline in "Figure 1", which shows chronological events in the case. I also show the \%-change in sales YoY (in "Figure 2"). 

By observing these diagrams in juxtaposition one can observe relationship between the geopolitical events mentioned in the timeline and dip in \%-change in sales. Between the years 1995 to year 2000 large amount of change in sales is observed and during the same time frame, as observed in the timeline from circle marked to that marked 6, one can observe negative coverage in media were reported. This shows the interdependence of consumer perception of a brand, the media agencies that influence this perception and phenomenon that how these factors interact with a business's performance. \\
\subsection{Timeline}
Figure 1: The Chronology of Events for IKEA's Child-Labor Ordeal \\

\begin{timeline}
\Task[1]{\textbf{Seed:Child Labor Deterrent Act} was proposed for debate in U.S. Congress\\\textbf{Year 1993} }
\Task[1.1]{\textbf{Unintended Consequences}}
\Task[2]{\textbf{Problem:}First report of Child Labor\\\textbf{Mitigation}\\Strategy:International Labor Organization (ILO)\\\textbf{Year 1994}}
\Task[2.1]{\textbf{Unintended Consequences 2} Induced unemployment in Pakistan}
\Task[3]{\textbf{Problem:}Monitoring was impractical\\\textbf{Scale and Nature of Manufacturing Process}\\\textbf{Fall 1994}}
\Task[4]{\textbf{May 1995}\\\textbf{Problem:}Investigative Report of Child Labor\\\textbf{Mitigation}\\Evidence:Report found fabricated}
\Task[5]{\textbf{Year 1998}\\Marianne Barner becomes IKEA\textquotesingle s children\textquotesingle s ombudsman}
\end{timeline}

\begin{timeline}
\Task[6]{\textbf{1998-99} 10-day educational tours of India\\ for IKEA's retail managers from different countries}
\Task[7]{\textbf{2004} registered success of joint program by UNICEF and IKEA\\ with ACL and Self-Help groups programs}
\Task[7.1]{\textbf{2004} IKEA terminated 10 supplier relationships for violations of its IWAY Code}
\Task[8]{\textbf{2005} Dahlvig's \textbf{10 Jobs for 10 Years} document released}
\end{timeline}